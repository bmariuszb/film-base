\documentclass{article}
\usepackage[utf8]{inputenc}
\usepackage[T1]{fontenc}
\usepackage[polish]{babel}

\title{Baza filmów}
\author{Autor: Mariusz Biegański}
\date{\today}

\begin{document}
\maketitle

\section{Opis funkcjonalności}
\subsection{Serwer}
\begin{itemize}
	\item Uruchomienie aplikacji w kontenerze Docker z użyciem pliku .war.
	\item Obsługa zdarzeń za pomocą języka AspectJ dla dodawania/aktualizowania oceny filmu.
	\item Wykorzystanie websocketów do wyświetlania formularza logowania.
	\item Generowanie aplikacji klienta.
	\item Generowanie HTML z tabelą z informacjami o filmach dostępnych w bazie danych.
	\item Generowanie HTML z formularzem logowania.
	\item Obsługa logowania.
	\item Generowanie HTML z formularzem rejestracji.
	\item Obsługa rejestracji.
	\item Obsługa wylogowania.
	\item Generowanie HTML z listą filmów do obejrzenia.
	\item Generowanie HTML z listą ocen filmów.
	\item Obsługa dodawania filmu do listy do obejrzenia.
	\item Obsługa usuwania filmu z listy do obejrzenia.
	\item Obsługa dodawania lub zmiany oceny filmu.
\end{itemize}

\subsection{Klient}
\begin{itemize}
\item Przesłanie zapytania o listę filmów.
\item Przesłanie zapytania o formularz logowania.
\item Przesłanie zapytania o formularz rejestracji.
\item Przesłanie zapytania o logowanie wraz z danymi.
\item Przesłanie zapytania o rejestrację wraz z danymi.
\item Przesłanie zapytania o wylogowanie.
\item Przesłanie zapytania o listę filmów do obejrzenia.
\item Przesłanie zapytania o usunięcie filmu z listy do obejrzenia.
\item Przesłanie zapytania o dodanie filmu do listy do obejrzenia.
\item Przesłanie zapytania o listę filmów wraz z oceną użytkownika.
\item Przesłanie zapytania o stworzenie/zmianę oceny filmu.
\end{itemize}

\subsection{Opis}
Serwer jest uruchamiany w kontenerze Docker z użyciem pliku .war. Obsługuje zdarzenia związane z dodawaniem i aktualizowaniem ocen filmów przy użyciu języka AspectJ. Wykorzystuje websockety do wyświetlania formularza logowania. Serwer generuje również aplikację klienta oraz różne HTML-owe widoki, takie jak tabela z informacjami o filmach, formularze logowania i rejestracji, listy filmów do obejrzenia oraz listy ocen filmów. \\
Klient komunikuje się z serwerem poprzez przesyłanie zapytań. Po otrzymaniu odpowiedzi od serwera, które zazwyczaj zawierają nową zawartość HTML, klient po prostu podmienia treść aktualnego dokumentu HTML na nową. W ten sposób klient dynamicznie aktualizuje wyświetlaną stronę, bez potrzeby przeładowywania całego dokumentu. \\
Dzięki takiemu podejściu serwer i klient współpracują, umożliwiając interakcję użytkownika z aplikacją filmową.

\section{Diagramy UML}
W ramach dokumentacji projektowej przedstawiam poniższe diagramy, które ilustrują różne aspekty mojej aplikacji filmowej:

%\begin{figure}[h]
%  \centering
%  \includegraphics[width=0.8\textwidth]{use-case.jpg}
%  \caption{Diagram przypadków użycia}
%\end{figure}
%
%\begin{figure}[h]
%  \centering
%  \includegraphics[width=0.8\textwidth]{class-diagram.jpg}
%  \caption{Diagram klas}
%\end{figure}
%
%\begin{figure}[h]
%  \centering
%  \includegraphics[width=0.8\textwidth]{sequence-diagram.jpg}
%  \caption{Diagram sekwencji}
%\end{figure}
%
%\begin{figure}[h]
%  \centering
%  \includegraphics[width=0.6\textwidth]{state-diagram.jpg}
%  \caption{Diagram stanów}
%\end{figure}

Te diagramy pomogą zrozumieć różne aspekty funkcjonalności i struktury aplikacji filmowej.

\section{Informacja uruchomieniowa}
Aby uruchomić aplikację Film-Base, postępuj zgodnie z poniższymi instrukcjami:

\begin{enumerate}
  \item Pobierz plik film-base.tar z repozytorium: \texttt{https://github.com/bmariuszb/film-base}.
  
  \item Załaduj obraz Docker z pliku film-base.tar, używając polecenia:
  \begin{verbatim}
  docker load -i film-base.tar
  \end{verbatim}
  
  \item Uruchom kontener Docker z aplikacją Film-Base, używając polecenia:
  \begin{verbatim}
  docker run --rm -p 127.0.0.1:8080:8080 --name film-base film-base
  \end{verbatim}
  
  \item Otwórz przeglądarkę internetową i przejdź do adresu:
  \begin{verbatim}
  http://localhost:8080/film-base-1.0-SNAPSHOT/api/jpa/movies
  \end{verbatim}
\end{enumerate}

Po wykonaniu tych kroków, powinieneś być w stanie uruchomić aplikację Film-Base i uzyskać dostęp do listy filmów poprzez podany adres URL.\\

Przykładowy użytkownik został stworzony z następującymi danymi logowania:
\begin{itemize}
  \item Nazwa użytkownika: admin
  \item Hasło: admin
\end{itemize}

Możesz użyć tych danych logowania, aby zalogować się do aplikacji i uzyskać dostęp do jej funkcjonalności.


\end{document}
